\documentclass[]{article}
\usepackage{amsmath}
\usepackage{amsfonts}
\usepackage{amssymb}
\usepackage{graphicx}
\usepackage{color}

\newcommand{\plain}{\color{black}}

\definecolor{c1}{RGB}{114,0,172}
\definecolor{c2}{RGB}{45,177,93}
\definecolor{c3}{RGB}{251,0,29}
\definecolor{c4}{RGB}{18,110,213}
\definecolor{c5}{RGB}{255,160,109}
\definecolor{c6}{RGB}{219,78,158}

\newcommand{\InitialIrradiance}{\color{c1}}
\newcommand{\wavelength}{\color{c2}}
\newcommand{\range}{\color{c3}}
\newcommand{\apertureDiameter}{\color{c4}}
\newcommand{\firstCrossing}{\color{c5}}
\newcommand{\displacement}{\color{c6}}

\title {Beam Expansion Analysis}
\author {Clay Rardin}

\begin{document}

\maketitle

\section{General Discussion of Beam Spreading for a Rectangular Aperture}
The diffraction angle $\Delta\theta$ of a laser beam of wavelength $\lambda$ going through an aperture of width D is given by $\dfrac{\lambda}{D}$.  The central maximum of the beam in the far-field is independent of distance between aperture and screen.  The spreading of the central maximum with length is given by \\
\begin{center}
$L \Delta\theta = \dfrac{2L\lambda}{D}$
\end{center}
We can remove the slit width and just assume a beam of width D and it will behave identically.  After collimation, a "parallel" beam of light spreads just as if it emerged from a single opening.

\subsection{Practice Problems}
\begin{enumerate}
\item Imagine a parallel beam of 546-nm light of width D = 0.5 mm propagating across the laboratory, a distance of 10 m.  Determine the final width of the beam due to diffraction spreading.
\paragraph{Solution}
 Using the equation $\dfrac{2L \lambda}{D}$ we get
\begin{center}
$W = \dfrac{2L\lambda}{D} = \dfrac{2 (10) (546 x 10^{-9})}{0.5 x 10^{-3}} = 21.8 mm $
\end{center}
\item Assume: 
\begin{itemize}
\item $I_{o}$ = 100 $W/cm^2$
\item A Square Aperture
\item Aperture Width $= 10$ cm
\item $\lambda = 1$ x $10^{-6}$ m
\item Range $= 1 $ km
\end{itemize}
Find the irradiance of the far-field beam 0.5 cm from the peak.\\
Using Matlab plot the shape of the beam from the peak to the zero crossing.
\paragraph{Solution}
Using the equation $\dfrac{2L \lambda}{D}$ we get
\begin{center}
$W = \dfrac{2L \lambda}{D} = \dfrac{2 (1x10^3)(1x10^{-6})}{1x10^{-1}} = $
\end{center}
\end{enumerate}

\section{General Discussion of Beam Spreading for a Circular Aperture}

\subsection{Practice Problems}

\end{document}
